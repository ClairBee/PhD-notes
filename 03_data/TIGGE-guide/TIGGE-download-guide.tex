\documentclass[10pt,fleqn]{article}
\usepackage{/home/clair/Documents/docstyle}

\usepackage{framed}
\usepackage{array}
\lstset{language = python,
	basicstyle = \ttfamily\footnotesize,
    commentstyle = \color{blue}\textit,
	keywordstyle = \color{purple},
    tabsize = 4,
    breakatwhitespace = true,
	showspaces = false,
    xleftmargin = .75cm,
	extendedchars = true}

\usepackage{pifont}
\newcommand{\xmark}{\ding{51}}
	
% If including  bibliography, configure quick build > PdfLaTeX + Bib(La)TeX + PdfLaTeX (x2) + view PDF

%----------------------------------------------------------------------
% reformat section headers to be smaller \& left-aligned
\titleformat{\section}{\normalfont\bfseries}{\thesection}{1em}{}
\titleformat{\subsection}{\normalfont\bfseries}{\thesubsection}{1em}{}
	
%----------------------------------------------------------------------

%\addtolength{\topmargin}{-0.5cm}
%\addtolength{\textheight}{1cm}

%======================================================================

\begin{document}


\section*{Getting started}

See the \href{https://software.ecmwf.int/wiki/display/WEBAPI/Access+ECMWF+Public+Datasets}{ECMWF guide to accessing public datasets} for detailed  instructions.

\textbf{Initial setup:}

\begin{itemize}

\item \href{https://apps.ecmwf.int/registration/}{Register} with the ECMWF

\item Once logged in, retrieve your \href{https://api.ecmwf.int/v1/key/}{API key} and paste it into \texttt{\$HOME/user/.ecmwfapirc}

\item Before downloading anything, you must \href{http://apps.ecmwf.int/datasets/data/interim-full-daily/licence/}{accept the terms and conditions}
\item Install the python \texttt{ecmwfapi} library \\[5pt]
\resizebox{0.95\textwidth}{!}{
	\begin{tabular}{l>{\ttfamily}l}
	Linux: & sudo apt-get install python-pip \\
			& sudo pip install https://software.ecmwf.int/wiki/download/attachments/56664858/ecmwf-api-client-python.tgz\\
	\end{tabular}
}
\end{itemize}

\vspace{10pt}
\textbf{Downloading data}:

Publicly available datasets are listed \href{http://apps.ecmwf.int/datasets/}{here}. Each has a GUI to select and download subsets of the available data (in .GRIB or .NETCDF format), or produce a python script for efficient batch download (`view the MARS request').

Your current job list can be viewed at \url{http://apps.ecmwf.int/webmars/joblist/}

MARS server activity (including the time taken to process other users' requests) can be viewed at \url{http://apps.ecmwf.int/mars-activity/} - unfortunately, this won't necessarily help to gauge how long your request is likely to spend in a queue, only how long it will take to process.\\[10pt]

\begin{framed}

\flushleft{\textbf{Size of available ensemble forecasts in TIGGE and S2S repositories}} \\[10pt]
% origin
\begin{tabular}{c>{\ttfamily}cccl}
& \textbf{origin}& \textbf{\href{https://software.ecmwf.int/wiki/display/TIGGE/Models}{TIGGE}} & \textbf{\href{https://software.ecmwf.int/wiki/display/S2S/Models}{S2S}} & \\
BoM & ammc &20 &$3\times11$ & Bureau of Meteorology (Aus) \\
CMA & babj &14 &4 & China Meteorological Association \\
CMC & cwao &20 &-& Meteorolocal Service of Canada \\
CNR-ISAC & isac &-&41 & Institute of Atmospheric Sciences \& Climate (Italy)\\
CNRM & lfpw &-&51 & National Centre for Meteorological Research (Fr)\\
CPTEC & sbsj &14 &-& Centro de Previs\~{a}o de Tempo e Estudos Clim\'{a}ticos (Br) \\
ECMWF & ecmf &50 &51 &  European Centre for Medium-range Weather Forecast \\
HMCR & rums &-&20 & Hydrometeorological Centre of Russia\\
JMA & rjtd &26 &25 & Japan Meteorological Agency \\
KMA & rksl & 23 & 4 & Korea Meteorological Administration\\
NCEP & kwbc &20 &16 & National Centers for Environmental Prediction (US) \\
UKMO & egrr &11 &4 & UK Met Office\\
\end{tabular}

\end{framed}


\newpage

\section*{Batch data retrieval using Python}
Some commands and parameters can only be accessed using batch processing, so writing a Python script directly may be necessary. A guide to all keywords can be found \href{https://software.ecmwf.int/wiki/display/UDOC/Identification+keywords#Identificationkeywords-class}{here}\\[7pt]

\begin{framed}
\textbf{General structure for download}
\begin{lstlisting}
#!/usr/bin/env python								# run python
from ecmwfapi import ECMWFDataServer				# download from ECMWF portal
server = ECMWFDataServer()							# 			"
    
server.retrieve({
    "class"     : "...",							# class must match dataset and stream
    "dataset"   : "...",
   	"stream"    : "....",							
    "type"      : "..",								# must be compatible with class, dataset, type
    "origin"	: "xxxx",							# forecasting centre (ensemble name)
    "number"	: "1/to/n",							# perturbed ensemble members to include (if applicable)
    "levtype"   : "...",							# atmospheric levels of interest
    "param"     : ".../.../.../..."					# parameters of interest
    "grid"      : "nn/nn",							# grid resolution (lat/long): must be integer fraction of 90
    "area"		: "70/-10/30/40"					# lat/long boundaries of area to extract: N/W/S/E
    "date"      : "yyyy-mm-dd/to/yyyy-mm-dd",		# date of analysis/forecast/observation
    "time"      : "00/06/12/18",					# analysis/forecast time in synoptic hours
    "step"      : "0",								# forecast time steps from forecast base time in hours (HH)
    "hdate"		: "19810206/19820206/19830206",		# hindcast base date (S2S only)
    "target"    : "filename.grib"					# target filename
})
\end{lstlisting}
\end{framed} % ERA-interim

% class & dataset
\begin{tabular}{>{\ttfamily}l>{\ttfamily}l>{\ttfamily}ll}
\textbf{\href{http://apps.ecmwf.int/codes/grib/format/mars/class/}{class}} & \textbf{dataset} & \textbf{\href{http://apps.ecmwf.int/codes/grib/format/mars/stream/}{stream}} & \\
ei & interim & oper & ERA-interim reanalysis data: atmospheric model \textit{(ie. weather)}\\
s2 & s2s & enfh & Subseasonal-to-seasonal data (up to 60 days): ensemble forecast hindcasts\\
ti & tigge & enfo & TIGGE medium-range forecasts (up to 14 days): ensemble prediction system\\
\end{tabular}

% type
\begin{tabular}{>{\ttfamily}cl}
\textbf{\href{http://apps.ecmwf.int/codes/grib/format/mars/type/}{type}} & \\
an & Analysis \\
cf & Control forecast\\
cl & Climatology \\
pf & Perturbed forecast\\
em & Ensemble mean\\
es & Ensemble SD\\
\end{tabular}

%levtype
\begin{tabular}{>{\ttfamily}cl>{\ttfamily}l}
\textbf{levtype} & & \textbf{levlist}\\
sfc & Surface & NA \\
pl & Pressure level (hPA) & 200/250/300/500/700/850/925/1000\\
\end{tabular}

% parameters
\begin{tabular}{>{\ttfamily}c>{\ttfamily}c>{\ttfamily}cll}
\textbf{\href{http://apps.ecmwf.int/codes/grib/param-db/}{ERA-Interim}} & \textbf{ECMWF} & \textbf{S2S}&\\
167.128 & 167 & &2m temperature & k\\
- & 172 && land-sea mask & (0-1) \\
228.128 &&& Total precipitation & (m) \\
165.128 &&& 10m U-component of wind & ms$^{-1}$\\
166.128 &&& 10m V-component of wind & ms$^{-1}$\\
151.128 &&& Mean sea level pressure & Pa\\
164.128 &&& Total cloud cover & (0-1) \\
\end{tabular}

\newpage

\begin{framed}
\textbf{Example of batch data retrieval into multiple files using Python}

\begin{lstlisting}
import time																# record time taken to process
from ecmwfapi import ECMWFDataServer
server = ECMWFDataServer()
 
start_date=['20140201']													# specify date range of interest
end_date=['20140228']
 
for s, e in zip(start_date, end_date):									# iterate over range
  try:
    server.retrieve({
      # specify source, parameters etc as usual
      'date'    : '{0}/to/{1}'.format(s, e),
      'target'  : 'cptec_{0}-{1}_T0_T360_00_12.grib'.format(s, e)		# download for each date in range
    })
 
  except Exception, err:
    print err
\end{lstlisting}
\end{framed}

\newpage

\section*{Examples of retrieval statements}

\begin{framed}
\textbf{\href{https://software.ecmwf.int/wiki/display/WEBAPI/Python+ERA-interim+examples}{ERA-interim:} \footnotesize{}}
\begin{lstlisting}   
server.retrieve({
    "stream"    : "oper",
    "levtype"   : "sfc",
    "param"     : "165.128/166.128/167.128",
    "dataset"   : "interim",
    "step"      : "0",
    "grid"      : "0.75/0.75",
    "time"      : "00/06/12/18",
    "date"      : "2014-07-01/to/2014-07-31",
    "type"      : "an",
    "class"     : "ei",
    "target"    : "interim_2014-07-01to2014-07-31_00061218.grib"
})
\end{lstlisting}
\end{framed} % ERA-interim

\begin{framed}
\textbf{\href{https://software.ecmwf.int/wiki/display/WEBAPI/Python+TIGGE+examples}{TIGGE:} \footnotesize{2m temperature on 01-Nov-2014, from ECMWF}}
\begin{lstlisting}
server.retrieve({
    "origin"    : "ecmf",
    "levtype"   : "sfc",
    "number"    : "1/to/50",							# specify ensemble members
    "expver"    : "prod",
    "dataset"   : "tigge",
    "step"      : "0/6/12/18",
    "grid"      : "0.5/0.5",
    "param"     : "167",
    "time"      : "00/12",
    "date"      : "2014-11-01",
    "type"      : "pf",									# perturbed forecast
    "class"     : "ti",
    "target"    : "tigge_2014-11-01_0012.grib"
})
\end{lstlisting}
\end{framed} % TIGGE


\begin{framed} % S2S
\textbf{\href{https://software.ecmwf.int/wiki/display/WEBAPI/Python+S2S+examples}{S2S:} \footnotesize{1 param (10m U wind) for all time steps, used to calibrate the 14-May-2015 real-time forecast}}
\begin{lstlisting}
server.retrieve({
    "class": "s2",
    "dataset": "s2s",
    "hdate": "2014-05-14",
    "date": "2015-05-14",
    "expver": "prod",
    "levtype": "sfc",
    "origin": "ecmf",
    "param": "165",
    "step": "0/to/1104/by/24",							# 46-day-ahead forecast
    "stream": "enfh",
    "target": "CHANGEME",
    "time": "00",
    "type": "cf",										# control forecast
})
\end{lstlisting}
\end{framed}


\end{document}
