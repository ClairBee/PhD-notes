\documentclass[10pt,fleqn]{article}
\usepackage{/home/clair/Documents/docstyle}

\usepackage{framed}
\usepackage{array}
\lstset{language = python,
	basicstyle = \ttfamily\footnotesize,
    commentstyle = \color{blue}\textit,
	keywordstyle = \color{purple},
    tabsize = 4,
    breakatwhitespace = true,
	showspaces = false,
    xleftmargin = .75cm,
	extendedchars = true}

\usepackage{pifont}
\newcommand{\xmark}{\ding{51}}
	
% If including  bibliography, configure quick build > PdfLaTeX + Bib(La)TeX + PdfLaTeX (x2) + view PDF

%----------------------------------------------------------------------
% reformat section headers to be smaller \& left-aligned
\titleformat{\section}{\normalfont\bfseries}{\thesection}{1em}{}
\titleformat{\subsection}{\normalfont\bfseries}{\thesubsection}{1em}{}
	
%----------------------------------------------------------------------

%\addtolength{\topmargin}{-0.5cm}
%\addtolength{\textheight}{1cm}

%======================================================================

\begin{document}


\section{Quick guide to downloading from ECMWF}

See the \href{https://software.ecmwf.int/wiki/display/WEBAPI/Access+ECMWF+Public+Datasets}{ECMWF guide to accessing public datasets} for detailed  instructions.

\textbf{Initial setup:}

\begin{itemize}

\item \href{https://apps.ecmwf.int/registration/}{Register} with the ECMWF

\item Once logged in, retrieve your \href{https://api.ecmwf.int/v1/key/}{API key} and paste it into \texttt{\$HOME/user/.ecmwfapirc}

\item Before downloading anything, you must \href{http://apps.ecmwf.int/datasets/data/interim-full-daily/licence/}{accept the terms and conditions}
\item Install the python \texttt{ecmwfapi} library \\[5pt]
\resizebox{0.95\textwidth}{!}{
	\begin{tabular}{l>{\ttfamily}l}
	Linux: & sudo apt-get install python-pip \\
			& sudo pip install https://software.ecmwf.int/wiki/download/attachments/56664858/ecmwf-api-client-python.tgz\\
	\end{tabular}
}
\end{itemize}

\vspace{10pt}
\textbf{Downloading data}:

Publicly available datasets are listed \href{http://apps.ecmwf.int/datasets/}{here}. Each has a GUI to select and download subsets of the available data (in .GRIB or .NETCDF format), or produce a python script for efficient batch download (`view the MARS request'). \textit{It is possible to download data from ECMWF in netcdf format; however, if multiple timesteps are required (as they usually are with forecast data), the grib-to-netcdf conversion performed during the download will fail, so it's easier just to download everything in .\texttt{grib} format and convert manually.}

Requests should be organised by date and time for maximum efficiency, picking up multiple parameters and steps simultaneously. Tempting though it may be to create a single download for all dates of interest, it will be much quicker to download a single day's forecasts at a time - requests are queued according to the number of tapes they need to access, so a single download for 4 months' data may have to wait for several hours before processing even begins, while 120 single-day forecasts covering the same period might be retrieved and downloaded within 15 minutes.
 
Your current job list can be viewed at \url{http://apps.ecmwf.int/webmars/joblist/}

MARS server activity (including the time taken to process other users' requests) can be viewed at \url{http://apps.ecmwf.int/mars-activity/} - unfortunately, this won't necessarily help to gauge how long your request is likely to spend in a queue, only how long it will take to process. The ECMWF efficiency website mentions that maintenance is usually performed on a Wednesday, which may affect the performance of the API server.

\newpage

\subsection{Batch retrieval using Python and R}
Create a Python script to download a single day's ensemble forecast from the TIGGE archive:

\begin{lstlisting}
#!/usr/bin/env python
	
import sys									# import 'sys' package to allow arguments to be passed from command line
from ecmwfapi import ECMWFDataServer
server = ECMWFDataServer()

dt = sys.argv[1]							# assign 'date' argument from command line

server.retrieve({
	"class"		: "ti",
	"dataset"	: "tigge",
	"date"		: "{0}".format(dt),												# 'date' variable 
	"expver"	: "prod",
	"grid"		: "1/1",														# resolution: 1 degree
	"area"		: "60/-6/50/2",													# limited to area surrounding UK
	"levtype"	: "sfc",
	"origin"	: "ecmf/egrr/kwbc/babj/cwao",									# list all ensembles
	"number"	: "1/to/50",													# capture all membbers of largest ensemble
	"param"		: "167",														# 167 = 2m temperature
	"step"		: "0/24/48/72/96/120/144/168/192/216/240/264/288/312/336/360",
	"time"		: "00:00:00/12:00:00",
	"type"		: "pf",
	"target"	: "./tigge/temp-pf-{0}.grib".format(dt,			# save file locally for greater efficiency
})
\end{lstlisting}


Now use R to iterate over all variables of interest, calling the Python script for each. The command \texttt{system} is the equivalent to running the first argument from the command line; the second argument, \texttt{wait = F}, submits all of the requests simultaneously, rather than waiting for each request to be completed before the next is downloaded.

\begin{lstlisting}[language = R]
require(rPython}
	          
invisible(sapply(c(2007:2016), function(yr) {											# iterate over all years
	dt.rng <- sapply(as.Date(paste0(yr, "-11-01")):as.Date(paste0(yr+1, "-02-28")),
                     function(dd) toString(as.Date(dd, origin = "1970-01-01")))
	invisible(sapply(dt.rng, function(dt) {												# iterate over all dates
	          system(paste('python tigge-t2m-24h-1d-request.py', dt), wait = F)
	}))
}))
\end{lstlisting}

\newpage
\subsection{Some useful retrieval codes}

\subsubsection{Size of available ensemble forecasts in TIGGE and S2S repositories}

\begin{tabular}{c>{\ttfamily}cccl}
& \textbf{origin}& \textbf{\href{https://software.ecmwf.int/wiki/display/TIGGE/Models}{TIGGE}} & \textbf{\href{https://software.ecmwf.int/wiki/display/S2S/Models}{S2S}} & \\
BoM & ammc &20 &$3\times11$ & Bureau of Meteorology (Aus) \\
CMA & babj &14 &4 & China Meteorological Association \\
CMC & cwao &20 &-& Meteorolocal Service of Canada \\
CNR-ISAC & isac &-&41 & Institute of Atmospheric Sciences \& Climate (Italy)\\
CNRM & lfpw &-&51 & National Centre for Meteorological Research (Fr)\\
CPTEC & sbsj &14 &-& Centro de Previs\~{a}o de Tempo e Estudos Clim\'{a}ticos (Br) \\
ECMWF & ecmf &50 &51 &  European Centre for Medium-range Weather Forecast \\
HMCR & rums &-&20 & Hydrometeorological Centre of Russia\\
JMA & rjtd &26 &25 & Japan Meteorological Agency \\
KMA & rksl & 23 & 4 & Korea Meteorological Administration\\
NCEP & kwbc &20 &16 & National Centers for Environmental Prediction (US) \\
UKMO & egrr &23 &4 & UK Met Office\\
\end{tabular}


\subsubsection{\href{http://apps.ecmwf.int/codes/grib/param-db/}{Parameters}}
\begin{tabular}{>{\ttfamily}c>{\ttfamily}c>{\ttfamily}cll}
\textbf{ERA-Interim} & \textbf{TIGGE} & \textbf{S2S}&\\
167.128 & 167 & &2m temperature & k\\
- & 172 && land-sea mask & (0-1) \\
228.128 &&& Total precipitation & (m) \\
165.128 &&& 10m U-component of wind & ms$^{-1}$\\
166.128 &&& 10m V-component of wind & ms$^{-1}$\\
151.128 &&& Mean sea level pressure & Pa\\
164.128 &&& Total cloud cover & (0-1) \\
\end{tabular}


\subsubsection{\href{http://apps.ecmwf.int/codes/grib/format/mars/type/}{Data type}}
% type
\begin{tabular}{>{\ttfamily}c>{\ttfamily}c>{\ttfamily}cl}
\textbf{ERA-Interim} & \textbf{TIGGE} & \textbf{S2S}&\\
an &    &    & Analysis \\
   & cf &    & Control forecast\\
   &    & cl & Climatology \\
   & pf &    & Perturbed forecast\\
   & em &    & Ensemble mean\\
   & es &    & Ensemble SD\\
\end{tabular}


%% class & dataset
%\begin{tabular}{>{\ttfamily}l>{\ttfamily}l>{\ttfamily}ll}
%\textbf{\href{http://apps.ecmwf.int/codes/grib/format/mars/class/}{class}} & \textbf{dataset} & \textbf{\href{http://apps.ecmwf.int/codes/grib/format/mars/stream/}{stream}} & \\
%ei & interim & oper & ERA-interim reanalysis data: atmospheric model \textit{(ie. weather)}\\
%s2 & s2s & enfh & Subseasonal-to-seasonal data (up to 60 days): ensemble forecast hindcasts\\
%ti & tigge & enfo & TIGGE medium-range forecasts (up to 14 days): ensemble prediction system\\
%\end{tabular}


%%levtype
%\begin{tabular}{>{\ttfamily}cl>{\ttfamily}l}
%\textbf{levtype} & & \textbf{levlist}\\
%sfc & Surface & NA \\
%pl & Pressure level (hPA) & 200/250/300/500/700/850/925/1000\\
%\end{tabular}




\newpage
\section{Reading downloaded data into R}

ECMWF data is stored in .\texttt{grib} format, which is very compact, but which does not play nicely in R. The \href{https://software.ecmwf.int/wiki/display/ECC/ecCodes+Home}
{\texttt{eccodes} package} provides a command-line tool to convert directly from grib to netcdf format; this is not currently available in any Linux repositories, so needs to be installed directly from the source files.



\subsection{Install \texttt{eccodes} to get conversion tools}
Download \href{https://software.ecmwf.int/wiki/display/ECC/Releases}{the latest tar.gz file}, then install following the directions given on the website. From the command line,

\begin{footnotesize}
	\begin{tabular}{>{\ttfamily}ll}
		\textit{sudo apt install cmake} & \textit{Install \texttt{cmake}, if not already installed}  \\
		tar -xzf  eccodes-2.2.0-Source.tar.gz & Unpack tarball  \\
		mkdir build; cd build & Create \& move to clean build directory  \\
		cmake & (all one command)\\
		$\-\ $ ../eccodes-2.2.0-Source & $\-\ $ Path to unpacked tarball \\
		$\-\ $ -DCMAKE\_INSTALL\_PREFIX=/usr/local  & $\-\ $ Set target directory\\
		$\-\ $ -DENABLE\_NETCDF=ON & $\-\ $ Enable \texttt{grib\_to\_netcdf} conversion\\
		$\-\ $ -DENABLE\_PYTHON=ON & $\-\ $ Offer Python interface \\
		
		make & Compile  \\
		ctest & Test  \\
		sudo make install & Install to specified directory  \\
	\end{tabular}
\end{footnotesize}



\subsection{Convert downloaded \texttt{.grib} data to \texttt{.nc} format}

Conversion from \texttt{.grib} to \texttt{.nc} is now quite straightforward. On the command line, run

\texttt{grib\_to\_netcdf <input-file>.grib -o <output-file>.nc -T}

Option \texttt{-T} is critical when downloading forecast data at multiple timesteps - this will cause the converter to treat the timesteps as a dimension (which it is), not as a variable (which causes the conversion to find multiple records for each dimension, and to crash)

\subsection{Import netcdf data into R}

A handy reference to importing netcdf data in R can be found at \url{http://geog.uoregon.edu/bartlein/courses/geog607/Rmd/netCDF_01.htm}. In short:

\begin{lstlisting}[language = R]
ncin <- nc_open("<converted-data>.nc")			# open the file

names(ncin$var); names(ncin$dim)				# list variables and dimensions
[1] "u10" "v10" "t2m"
[1] "longitude" "latitude"  "number"    "step"      "time"      "date"

# get dimensions
lon <- ncvar_get(ncin, "longitude")         
lat <- ncvar_get(ncin, "latitude")
em <- ncvar_get(ncin, "number")                                         # ensemble members
lt <- ncvar_get(ncin, "step")                                           # time steps (look-ahead time)
f.time <- ncvar_get(ncin, "time")                                       # times that forecast was made
f.date <- as.Date(ncvar_get(ncin, "date"), origin = "1900-01-01")       # dates that forecast was made

# extract any variable as an array with the given dimensions
temp <- ncvar_get(ncin, "t2m")

nc_close(ncin)									# close the file (otherwise it will stay open in your workspace) 
\end{lstlisting}


%\subsection{Installing the \texttt{gribr} package}
%
%A user has very helpfully produced and shared an R package to import .grib files into R: \url{https://github.com/nawendt/gribr}\\[10pt]
%
%\textbf{First, install ecCodes.} 
%
%\textit{In Ubuntu, setup should be more straightforward if ecCodes is installed to a folder within the default system file path. This may differ among versions and installations, but seems to default to \texttt{/usr/local/lib}.\\
%\texttt{gedit /etc/ld.so.conf.d/libc.conf} on the command line (or some other text editor in place of \texttt{gedit}) should display the default path. }
%
%
%
%Set source \& target directories  \\
%
%\textbf{Then install the \texttt{gribr} package} directly from GitHub. Despite installing a folder on the \texttt{ld} path, I needed to create an \texttt{.Renviron} file in \texttt{/home}, containing the following:
%\begin{verbatim}
%ECCODES_LIBS=/usr/local/lib/eccodes/lib
%ECCODES_CPPFLAGS=/usr/local/lib/eccodes/include
%LD_LIBRARY_PATH=<current library path>:/usr/local/lib/eccodes/lib
%\end{verbatim}
%
%\vspace{12pt}
%
%To find \texttt{<current library path>}, in RStudio run \texttt{Sys.getenv("LD\_LIBRARY\_PATH")}, then copy and paste the string into the \texttt{.Renviron} file. 
%
%You will need to close and reopen RStudio for the new environment variables to be loaded on startup; it should now be possible to install \texttt{gribr} and its dependencies:
%\begin{verbatim}
%install.packages("proj4")
%devtools::install_github("nawendt/gribr", 
%                         args = "--configure-args='ECCODES_LIBS=-L/usr/local/lib/eccodes/lib
%                                                   ECCODES_CPPFLAGS=-I/usr/local/lib/eccodes/include'")
%\end{verbatim}





\end{document}

\newpage
\subsection{Batch data retrieval using Python}
Some commands and parameters can only be accessed using batch processing, so writing a Python script directly may be necessary. A guide to all keywords can be found \href{https://software.ecmwf.int/wiki/display/UDOC/Identification+keywords#Identificationkeywords-class}{here}\\[7pt]

\begin{framed}
\textbf{General structure for download}
\begin{lstlisting}
#!/usr/bin/env python								# run python
from ecmwfapi import ECMWFDataServer				# download from ECMWF portal
server = ECMWFDataServer()							# 			"
    
server.retrieve({
    "class"     : "...",							# class must match dataset and stream
    "dataset"   : "...",
   	"stream"    : "....",							
    "type"      : "..",								# must be compatible with class, dataset, type
    "origin"	: "xxxx",							# forecasting centre (ensemble name)
    "number"	: "1/to/n",							# perturbed ensemble members to include (if applicable)
    "levtype"   : "...",							# atmospheric levels of interest
    "param"     : ".../.../.../..."					# parameters of interest
    "grid"      : "nn/nn",							# grid resolution (lat/long): must be integer fraction of 90
    "area"		: "70/-10/30/40"					# lat/long boundaries of area to extract: N/W/S/E
    "date"      : "yyyy-mm-dd/to/yyyy-mm-dd",		# date of analysis/forecast/observation
    "time"      : "00/06/12/18",					# analysis/forecast time in synoptic hours
    "step"      : "0",								# forecast time steps from forecast base time in hours (HH)
    "hdate"		: "19810206/19820206/19830206",		# hindcast base date (S2S only)
    "target"    : "filename.grib"					# target filename
})
\end{lstlisting}
\end{framed} % ERA-interim



\newpage

\begin{framed}
\textbf{Example of batch data retrieval into multiple files using Python}

\begin{lstlisting}
import time																# record time taken to process
from ecmwfapi import ECMWFDataServer
server = ECMWFDataServer()
 
start_date=['20140201']													# specify date range of interest
end_date=['20140228']
 
for s, e in zip(start_date, end_date):									# iterate over range
  try:
    server.retrieve({
      # specify source, parameters etc as usual
      'date'    : '{0}/to/{1}'.format(s, e),
      'target'  : 'cptec_{0}-{1}_T0_T360_00_12.grib'.format(s, e)		# download for each date in range
    })
 
  except Exception, err:
    print err
\end{lstlisting}
\end{framed}

\newpage

\subsection{Examples of retrieval statements}

\begin{framed}
\textbf{\href{https://software.ecmwf.int/wiki/display/WEBAPI/Python+ERA-interim+examples}{ERA-interim:} \footnotesize{}}
\begin{lstlisting}   
server.retrieve({
    "stream"    : "oper",
    "levtype"   : "sfc",
    "param"     : "165.128/166.128/167.128",
    "dataset"   : "interim",
    "step"      : "0",
    "grid"      : "0.75/0.75",
    "time"      : "00/06/12/18",
    "date"      : "2014-07-01/to/2014-07-31",
    "type"      : "an",
    "class"     : "ei",
    "target"    : "interim_2014-07-01to2014-07-31_00061218.grib"
})
\end{lstlisting}
\end{framed} % ERA-interim

\begin{framed}
\textbf{\href{https://software.ecmwf.int/wiki/display/WEBAPI/Python+TIGGE+examples}{TIGGE:} \footnotesize{2m temperature on 01-Nov-2014, from ECMWF}}
\begin{lstlisting}
server.retrieve({
    "origin"    : "ecmf",
    "levtype"   : "sfc",
    "number"    : "1/to/50",							# specify ensemble members
    "expver"    : "prod",
    "dataset"   : "tigge",
    "step"      : "0/6/12/18",
    "grid"      : "0.5/0.5",
    "param"     : "167",
    "time"      : "00/12",
    "date"      : "2014-11-01",
    "type"      : "pf",									# perturbed forecast
    "class"     : "ti",
    "target"    : "tigge_2014-11-01_0012.grib"
})
\end{lstlisting}
\end{framed} % TIGGE


\begin{framed} % S2S
\textbf{\href{https://software.ecmwf.int/wiki/display/WEBAPI/Python+S2S+examples}{S2S:} \footnotesize{1 param (10m U wind) for all time steps, used to calibrate the 14-May-2015 real-time forecast}}
\begin{lstlisting}
server.retrieve({
    "class": "s2",
    "dataset": "s2s",
    "hdate": "2014-05-14",
    "date": "2015-05-14",
    "expver": "prod",
    "levtype": "sfc",
    "origin": "ecmf",
    "param": "165",
    "step": "0/to/1104/by/24",							# 46-day-ahead forecast
    "stream": "enfh",
    "target": "CHANGEME",
    "time": "00",
    "type": "cf",										# control forecast
})
\end{lstlisting}
\end{framed}